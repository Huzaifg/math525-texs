We make use of Corollary 4.2. Consider an arbitrary primal and its dual:

\begin{align}
    \text{minimize } &c'x &&\notag\\
    \text{subject to } &a_i'x\geq b_i &&  i\in M_1 \notag\\
                    &a_i'x\leq b_i &&  i \in M_2 \notag\\
                    &a_i'x= b_i &&  i \in M_3 \notag\\
                    &x_j\geq 0 && j \in N_1 \notag\\
                    &x_j\leq 0 && j \in N_2 \notag\\
                    &x_j\text{ free} && j \in N_3 \notag\\\\  
    \text{maximize } &p'b\notag&&\\
    \text{subject to } &p_i\geq 0 &&  i\in M_1 \notag\\
                    &p_i\leq 0 &&  i \in M_2 \notag\\
                    &p_i\text{ free} &&  i \in M_3 \notag\\
                    &p'A_j\leq c_j && j \in N_1 \notag\\
                    &p'A_j\geq c_j && j \in N_2 \notag\\
                    &p'A_j=c_j && j \in N_3 \notag
\end{align}

By the problem statement, any LP problem that we consider must have some optimal $x$. By strong duality, we know that its dual has some optimal $p$ such that $c'x=p'b$.

To find some optimal solution $x$, we then pass the following system of inequalities to the oracle:

\begin{align}
					&c'x-p'b=0\notag\\\\
                    &a_i'x\geq b_i &&  i\in M_1 \notag\\
                    &a_i'x\leq b_i &&  i \in M_2 \notag\\
                    &a_i'x= b_i &&  i \in M_3 \notag\\
                    &x_j\geq 0 && j \in N_1 \notag\\
                    &x_j\leq 0 && j \in N_2 \notag\\\\
                    &p_i\geq 0 &&  i\in M_1 \notag\\
                    &p_i\leq 0 &&  i \in M_2 \notag\\
                    &p'A_j\leq c_j && j \in N_1 \notag\\
                    &p'A_j\geq c_j && j \in N_2 \notag\\
                    &p'A_j=c_j && j \in N_3 \notag
\end{align}

If the oracle then returns a solution $(x, p)$ to this system of inequalities, we will know the following:

\begin{itemize}
\item $x$ is a feasible solution to the primal, as it satisfies $a_i'x\geq b_i$ for all $i\in M_1$, $a_i'x\leq b_i$ for all $i\in M_2$, and $a_i'x=b_i$ for all $i\in M_3$, as well as $x_j\geq 0$ for all $j\in N_1$ and $x_j\leq 0$ for all $j\in N_2$.
\item $p$ is a feasible solution to the dual, as it satisfies $p'A_j\leq c_j$ for all $j\in N_1$, $p'A_j\geq c_j$ for all $j\in N_2$, and $p'A_j=c_j$ for all $j\in N_3$, as well as $p_i\geq 0$ for all $i\in M_1$ and $p_i\leq 0$ for all $i\in M_2$.
\item $c'x=p'b$, since $x$ and $p$ satisfy the equality $c'x-p'b=0$.
\end{itemize}

Therefore, since we have feasible solutions $x$ and $p$ to the primal and dual respectively, where $c'x=p'b$, $x$ is an optimal solution to the primal problem.

We now, by contradiction, will show that the oracle cannot state that no solution exists to our system. Suppose that oracle states that no such solution $(x, p)$ exists. However, we know by the problem statement that there exists some optimal solution $x^*$ to the primal, and strong duality tells us that there must then exist some solution $p^*$ to the dual where $c'x^*=p'b^*$. Therefore $(x^*, p^*)$ is a solution to this system of equations, which contradicts the oracle's statement that there are no solutions. Therefore, the oracle cannot state that no solution exists, and we will always receive a solution by providing the oracle with this system of equations.

Q.E.D.