We show that exactly one of the following two constraints must hold:

\begin{enumerate}
\item There exists some $x\neq 0$ such that $Ax=0, x\geq 0$.
\item There exists some $p$ such that $p'A>0'$.
\end{enumerate}

In both proofs, we consider the following primal and dual problem, where $c>0$:\\

\underline{Primal}
\begin{align}
    \text{minimize } &0'p \notag\\
    \text{subject to } &p'A\geq c\notag\\\\
\end{align}

\underline{Dual}
\begin{align}
    \text{maximize } &c'x\notag\\
    \text{subject to } &A'x=0\notag\\
                    &x\geq 0\notag\\
\end{align}

\underline{(1) implies $\neg$(2)}

Suppose that there exists some $x\neq 0$ such that $Ax=0$ and $x\geq 0$. Since $x$ satisfies those conditions, $x$ is a feasible solution to the dual, with cost $c'x$. Since $c>0$ and $x\geq 0$, we know $c'x\geq 0$. But, since $x\neq 0$, there exists some $x_i>0$, for some $1\leq i\leq n$. Thus $c_ix_i>0$ and $c'x>0$.

Consider the vector $2x$. We have $A(2x)=2(Ax)=0$, and $2x\geq 0$. Thus $2x$ is also a feasible solution to the dual, with cost $c'(2x)=2(c'x)$. Thus, for any feasible solution, we can find another optimal solution with twice the cost. Since we have a solution $x$ with $c'x>0$, we can continue to iterate in this way to show that the dual is unbounded.

Since the dual is unbounded, by Corollary 4.1 the primal is infeasible, \textbf{for any $c>0$}. If there existed some $p$ such that $p'A>0$, there would be some $c$ such that the primal was feasible. Thus there can be no such $p$.

\underline{(2) implies $\neg$(1)}

Suppose there exists some $p$ such that $p'A>0'$. In that case, there exists at least one $c>0$ such that the primal is feasible, and the cost of all solutions to the primal is $0$. By strong duality, the dual must also be feasible, with optimal cost $0$. It is obvious that $x=0'$ is an optimal feasible solution for the dual, with $c'x=0$.

Suppose, to reach a contradiction, that there did exist some $x^*\neq 0$ such that $Ax^*=0, x^*\geq 0$. In that case, $x^*$ would also be a solution to the dual, with optimal cost $c'x^*\geq 0$. Since $c>0$ and $x^*\neq 0$, there exists some index $i$ such that $x^*_i>0$, and $c'x^*>0$. However, this would violate the principle of strong duality, which states that the optimal cost of the dual is equal to the optimal cost of the primal (namely $0$). Thus there cannot exist some $x^*$ such that $Ax^*=0, x^*>0$, i.e. (1) cannot hold.

Q.E.D.