% The given problem can be rewritten in matrix form as,

% \begin{align}
%     \text{min } &\text{max } Ax - b \notag\\
%     \text{s.t }  &x \in \mathbb{R}^n\notag\\
% \end{align}

% where $A$ is a matrix made up of the $m$ rows $a_i'$. \\

The problem represents a minimization of a piecewise linear convex objective function that we dealt with in chapter 1. Using slide 90 of 140 in chapter 1, we can convert this into a LP as;

\begin{align}
    \text{min } &z \notag\\
    \text{s.t }  &z \ge a_i'x - b_i \text{ for } i=1,...m\notag
\end{align}

We can concatenate these constraints into a matrix $A\in\mathbb{R}^{m\times n}$, also making use of the vector $1'\in\mathbb{R}^m$ of ones:

\begin{align}
    \text{min } &z \notag\\
    \text{s.t }  &1'z \ge Ax - b \notag
\end{align}

Rearranging this we get,

\begin{align}
    \text{min } &z \notag\\
    \text{s.t }  &-Ax + 1'z \ge -b  \notag\\
                &x, z \text{ are free }\notag
\end{align}

Now the dual of this is, 

\begin{align}
    \text{min } &p'(-b) \notag\\
    \text{s.t }  &p \ge 0 \notag\\
                &-p'A = 0 \notag\\
                &p'1' = 1 \notag
\end{align}

We note that if $-p'A=0$, then $p'A=0$, yielding:

\begin{align}
    \text{min } &p'(-b) \notag\\
    \text{s.t }  &p \ge 0 \notag\\
                &p'A = 0 \notag\\
                &p'1' = 1 \notag
\end{align}

Notice that $p'1' = p' [1, 1, .....1]' = \sum_{i=1}^{m} p_i$. Thus, the problem is written as:

\begin{align}
    \text{min } &p'(-b) \notag\\
    \text{s.t }  &p \ge 0  \notag\\
                &p'A = 0 \notag\\
                &\sum_{i=1}^{m} p_i = 1 \notag
\end{align}

(a) If $p$ satisfies the constraints $p'A=0', \sum_{i=1}^m p_i=1, p\geq 0$, then it is a feasible solution to the dual problem shown above. Weak duality tells us that the cost $(-p'b)$ of any feasible solution $p$ to the dual must be less than the optimal cost of the primal (namely $v$). Thus, for any $p$, $(-p'b)\leq v$.\\

(b) We have already shown that the LP given in the question is the dual of the original LP. From the strong duality theorem, we know that if the primal has an optimal solution (which it must, to have a finite optimal cost $v$) then the dual has an optimal solution as well, with the same optimal cost. Thus, the optimal cost of the given LP is $v$.