Considering the given primal LP, we can construct the dual LP using the rules prescribed in the slides as; \\

\begin{align}
    \text{maximize } &p'c \notag\\
    \text{subject to } &p \ge 0 \notag\\
                    &p'A \le c \notag
\end{align}

The goal is to show $x^*$ is an optimal solution. This is done by showing that a solution (proved to be feasible below) $p^* = x^*$ to the dual produces the same cost as $x^*$ to the primal. From the strong duality theorem, we know that the optimal costs of the primal and dual are equal. We can use this fact to argue that if $x^*$ and $p^*$ produce the same cost, then the cost is optimal and $x^* = p^*$ are the optimal solutions.

Thus, consider $x^*$ to be a feasible solution to the primal LP and let $p^* = x^*$. \\
Then $p^*$ is feasible for the dual LP because;
\[ p^* = x^* \ge 0\]
Using the fact that $A = A'$ as $A$ is symmetric and square, 
\[Ax^* = Ap^* = A'p^* = (p^*)'A = c \text{ thus the inequality is satisfied. } p^* \text{ is thus feasible. }\]
Additionally, writing the costs in terms of $x^*$ and $p^*$
\[c'x^* = (x^*)'c \text{ as a scalar and its transpose are equal}\]
Since, $x^* = p^*$
\[(x^*)'c = (p^*)'c\]
Thus, the primal and dual have the same cost for $x^* = p^*$. Thus, based on the argument above, this is the optimal cost and $x^*$ is the optimal solution.


