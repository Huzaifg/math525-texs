\documentclass[11pt]{article}
\usepackage{amsfonts,amsmath,amssymb,amsthm} % Math packages
\usepackage{graphicx} % For including images
\usepackage{titlesec} % For customizing section titles
\usepackage{titling} % For customizing the title
\usepackage{geometry} % For adjusting page margins
\usepackage{datetime} % For formatting the date
\usepackage{hyperref} % For adding hyperlinks


% Customize the title format
\pretitle{\begin{center}\Large\bfseries}
\posttitle{\par\end{center}}
\preauthor{\begin{center}\large}
\postauthor{\end{center}}
\predate{\begin{center}\large}
\postdate{\par\end{center}}

% Customize the section title format
\titleformat{\section}{\large\bfseries}{\thesection.}{0.5em}{}

% Adjust the page margins
\geometry{margin=1in}

\begin{document}

% Create the title page
\begin{titlepage}
    \centering
    \vspace*{0.5cm}
    \par\normalfont\fontsize{35}{35}\sffamily\selectfont
    Assignment 03 - MATH 525\par 
    \vspace*{1cm}
    {\huge\bfseries Huzaifa Unjhawala \\  Gabe Selzer \par} 
    \vspace*{1cm}
    {\Large\itshape Date: \today\par} 
    \vfill
\end{titlepage}

\tableofcontents
\newpage

Problem 2

We note that Theorem 2.6 from the lecture notes states that a polyhedron contains an extreme point if and only if it contains a line.

Standard form problems cannot possibly contain a line, as they are confined as a subset of the positive orthant $\{x|x\geq 0\}$. This is not a requirement of general form problems, and as such they can contain a line. For example, the polyhedron $P=\{x\in\mathbb{R}^2|x_1\geq 4\}$ comprises a half-space, containing many lines (such as $x_1=5$).

Just because we can turn a general form polyhedron into a standard form polyhedron with the same optimal cost, this does not mean that the polyhedron is equivalent. By transforming a polyhedron into standard form, we add dimensions such that we can maintain the constraints of the original problem through a polyhedron in the positive orthant. The fact that the general form polyhedron and the standard form polyhedron (may) lie in different dimensionalities is enough to reason that they may not be equivalent, despite leading to an equivalent LP problem. For this reason, we cannot conclude that every nonempty polyhedron contains an extreme point.

\end{document}