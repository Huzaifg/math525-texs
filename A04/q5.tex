The tableau is currently, 


$$
\begin{array}{c}
\\
 \\
x_3 \\
x_4 \\ 
x_5 \\
\end{array}
\begin{array}{|c|ccccc|}
    \hline
    & x_1 & x_2 & x_3 & x_4 & x_5\\ \hline
  -10 & \delta & -2 & 0 & 0 & 0\\ \hline
  4 & -1 & \eta & 1 & 0 & 0\\
  1 & \alpha & -4 & 0 & 1 & 0\\
  \beta & \gamma & 3 & 0 & 0 & 1\\\hline
\end{array}
$$

\subsection*{a}
The question asks for values of the unknowns to be such that the current solution is optimal and there are multiple optimal solutions. \\

We can see that $\bar{c}_2$ is negative. Therefore, for the current basic feasible solution to be optimal, we need a degenerate basic feasible solution (Theorem 3.1). Thus, we set $\beta$ to $0$ and pivot $x_2$ to enter the basis with $x_5$ leaving the basis. We choose $\alpha = 5$, $\gamma = -3$, $\delta = 2$ and $\eta = -2$ so that we avoid having $x_5$ enter the basis again. After the first pivot, we will get the tableau,

$$
\arraycolsep=1.4pt\def\arraystretch{2.2}
\begin{array}{c}
\\
 \\
x_3 \\
x_4 \\ 
x_2 \\
\end{array}
\arraycolsep=4.4pt\def\arraystretch{2.2}
\begin{array}{|c|ccccc|}
    \hline
    & x_1 & x_2 & x_3 & x_4 & x_5\\ \hline 
 -10 & 0 & 0 & 0 & 0 & \dfrac{2}{3}\\ \hline
  4 & -3 & 0 & 1 & 0 & \dfrac{2}{3}\\
  1 & 1 & 0 & 0 & 1 & \dfrac{4}{3}\\
  0 & -1 & 1 & 0 & 0 & \dfrac{1}{3}\\\hline
\end{array}
$$

The cost still remains at $-10$ and all the $c_j$'s are positive. Thus, $x = (0,0,4,1,0)$ is optimal. However this is the same point as we had before the pivot, so we have not yet proved that there are "multiple" optimal solutions. \\

Since $x_1$ is a nonbasic variable, however, we will now pivot $x_1$ and bring it to the basis. We can only choose the 2$^{nd}$ row as the pivot row as it is the only positive $u_j$. Thus, $x_1$ enters the basis with $\theta^* = 1$ and $x_4$ exits the basis. We then get the following tableau

$$
\arraycolsep=1.4pt\def\arraystretch{2.2}
\begin{array}{c}
\\
 \\
x_3 \\
x_1 \\ 
x_2 \\
\end{array}
\arraycolsep=4.4pt\def\arraystretch{2.2}
\begin{array}{|c|ccccc|}
    \hline
    & x_1 & x_2 & x_3 & x_4 & x_5\\ \hline 
 -10 & 0 & 0 & 0 & 0 & \dfrac{2}{3}\\ \hline
  7 & 0 & 0 & 1 & 3 & \dfrac{14}{3}\\
  1 & 1 & 0 & 0 & 1 & \dfrac{4}{3}\\
  1 & 0 & 1 & 0 & 1 & \dfrac{5}{3}\\\hline
\end{array}
$$

The cost still remains at $-10$ and all the $c_j$'s are positive. Thus the basic feasible solution,$x = (1,1,7,0,0)$ is of the same cost, making it optimal as well. Thus we have multiple optimal basic feasible solutions for this choice of unknown parameters.



\subsection*{b}
The optimal cost is $- \infty$ if we choose a $j$ for which $\bar{c}_j < 0$ and all of the $u$'s in column $j$ are positive. Based on the available unknowns, we can only make column $1$ such that all the $u$'s are positive. We thus choose $\boldsymbol{\delta < 0}$ (say $-2$) and ensure that we choose the minimum index pivoting rule. Then we set $\boldsymbol{\alpha \le 0 \text{ and } \gamma \le 0}$ (say $\alpha = -2$ and $\gamma = -2$). We also set $\beta > 0 $ (say 2) to ensure the solution is feasible and choose a random value of 2 for $\eta$. The tableau is thus, 

$$
\begin{array}{c}
\\
 \\
x_3 \\
x_4 \\ 
x_5 \\
\end{array}
\begin{array}{|c|ccccc|}
    \hline
    & x_1 & x_2 & x_3 & x_4 & x_5\\ \hline
  -10 & -2 & -2 & 0 & 0 & 0\\ \hline
  4 & -1 & 2 & 1 & 0 & 0\\
  1 & -2 & -4 & 0 & 1 & 0\\
  2 & -2 & 3 & 0 & 0 & 1\\\hline
\end{array}
$$
Then, when we attempt to bring $x_1$ into the basis, we see that the optimal cost is $- \infty$ as all the $d$'s are negative.

\subsection*{c}
The current solution is not optimal if it is non-degenerate since we have a $\bar{c}_j < 0$. Thus by setting $\boldsymbol{\beta > 0}$ (say 2), we obtain a feasible but not optimal solution. We can choose any random values for the other unknowns. Thus we use the below tableau,


$$
\begin{array}{c}
\\
 \\
x_3 \\
x_4 \\ 
x_5 \\
\end{array}
\begin{array}{|c|ccccc|}
    \hline
    & x_1 & x_2 & x_3 & x_4 & x_5\\ \hline
  -10 & 10 & -2 & 0 & 0 & 0\\ \hline
  4 & -1 & 10 & 1 & 0 & 0\\
  1 & -5 & -4 & 0 & 1 & 0\\
  2 & 1 & 3 & 0 & 0 & 1\\\hline
\end{array}
$$
