The tableau is currently, 


$$
\begin{array}{c}
\\
 \\
x_3 \\
x_4 \\ 
x_5 \\
\end{array}
\begin{array}{|c|ccccc|}
    \hline
    & x_1 & x_2 & x_3 & x_4 & x_5\\ \hline
  -10 & \delta & -2 & 0 & 0 & 0\\ \hline
  4 & -1 & \eta & 1 & 0 & 0\\
  1 & \alpha & -4 & 0 & 1 & 0\\
  \beta & \gamma & 3 & 0 & 0 & 1\\\hline
\end{array}
$$

\subsection*{a}
The question asks for values of the unknowns to be such that the current solution is optimal and there are multiple optimal solutions. \\

We can see that $\bar{c}_2$ is negative. Therefore, for the current basic feasible solution to be optimal, we need a degenerate basic feasible solution (Theorem 3.1). Thus, we set $\beta$ to $0$ and pivot $x_2$ to enter the basis with $x_5$ leaving the basis. We choose $\alpha = 5$, $\gamma = -3$, $\delta = 2$ and $\eta = -2$ so that we avoid having $x_5$ enter the basis again. 

$$
\begin{array}{c}
\\
 \\
x_3 \\
x_4 \\ 
x_5 \\
\end{array}
\begin{array}{|c|ccccc|}
    \hline
    & x_1 & x_2 & x_3 & x_4 & x_5\\ \hline
  -10 & 2 & -2 & 0 & 0 & 0\\ \hline
  4 & -1 & -2 & 1 & 0 & 0\\
  1 & 5 & -4 & 0 & 1 & 0\\
  0 & -3 & 3 & 0 & 0 & 1\\\hline
\end{array}
$$


After the first pivot, with $x_2$ entering the basis and $x_5$ leaving the basis, we will get the tableau,

$$
\arraycolsep=1.4pt\def\arraystretch{2.2}
\begin{array}{c}
\\
 \\
x_3 \\
x_4 \\ 
x_2 \\
\end{array}
\arraycolsep=4.4pt\def\arraystretch{2.2}
\begin{array}{|c|ccccc|}
    \hline
    & x_1 & x_2 & x_3 & x_4 & x_5\\ \hline 
 -10 & 0 & 0 & 0 & 0 & \dfrac{2}{3}\\ \hline
  4 & -3 & 0 & 1 & 0 & \dfrac{2}{3}\\
  1 & 1 & 0 & 0 & 1 & \dfrac{4}{3}\\
  0 & -1 & 1 & 0 & 0 & \dfrac{1}{3}\\\hline
\end{array}
$$

The cost still remains at $-10$ and all the $c_j$'s are positive. Thus, $x = (0,0,4,1,0)$ is optimal. However this is the same point as we had before the pivot, so we have not yet proved that there are "multiple" optimal solutions. \\

Since $x_1$ is a nonbasic variable, however, we will now pivot $x_1$ and bring it to the basis. We can only choose the 2$^{nd}$ row as the pivot row as it is the only positive $u_j$. Thus, $x_1$ enters the basis with $\theta^* = 1$ and $x_4$ exits the basis. We then get the following tableau

$$
\arraycolsep=1.4pt\def\arraystretch{2.2}
\begin{array}{c}
\\
 \\
x_3 \\
x_1 \\ 
x_2 \\
\end{array}
\arraycolsep=4.4pt\def\arraystretch{2.2}
\begin{array}{|c|ccccc|}
    \hline
    & x_1 & x_2 & x_3 & x_4 & x_5\\ \hline 
 -10 & 0 & 0 & 0 & 0 & \dfrac{2}{3}\\ \hline
  7 & 0 & 0 & 1 & 3 & \dfrac{14}{3}\\
  1 & 1 & 0 & 0 & 1 & \dfrac{4}{3}\\
  1 & 0 & 1 & 0 & 1 & \dfrac{5}{3}\\\hline
\end{array}
$$

The cost still remains at $-10$ and all the $c_j$'s are positive. Thus the basic feasible solution,$x = (1,1,7,0,0)$ is of the same cost, making it optimal as well. Thus we have multiple optimal basic feasible solutions for this following choice of parameters:

$$
\alpha=5, \beta=0, \delta=2, \eta=-2, \gamma=-3
$$

\subsection*{b}
The optimal cost is $- \infty$ if we choose a $j$ for which $\bar{c}_j < 0$ and all of the $u$'s in column $j$ are positive. Based on the available unknowns, we can only make column $1$ such that all the $u$'s are positive. We thus choose $\boldsymbol{\delta < 0}$ (say $-2$). Then we set $\boldsymbol{\alpha \le 0 \text{ and } \gamma \le 0}$ (say $\alpha = -2$ and $\gamma = -2$). We also set $\beta > 0 $ (say 2) to ensure the solution is feasible. Choosing $\eta$ is important because if we end up pivoting with column 2, we should choose an $\eta$ that gives us negative $u$'s for all the negative reduced costs. Choosing $\eta$ as 8 satisfies this requirement. Consider starting with the tableau, 

$$
\begin{array}{c}
\\
 \\
x_3 \\
x_4 \\ 
x_5 \\
\end{array}
\begin{array}{|c|ccccc|}
    \hline
    & x_1 & x_2 & x_3 & x_4 & x_5\\ \hline
  -10 & -2 & -2 & 0 & 0 & 0\\ \hline
  4 & -1 & 8 & 1 & 0 & 0\\
  1 & -2 & -4 & 0 & 1 & 0\\
  2 & -2 & 3 & 0 & 0 & 1\\\hline
\end{array}
$$
With this tableau, we have $\bar{c}_1<0$ and $\bar{c}_2<0$. When we attempt to bring $x_1$ into the basis, we see that the optimal cost is $- \infty$ as all the $d$'s are negative. If we attempt to bring $x_2$ to the basis, we get the following tableau,

$$
\begin{array}{c}
\\
 \\
x_2 \\
x_4 \\ 
x_5 \\
\end{array}
\begin{array}{|c|ccccc|}
    \hline
    & x_1 & x_2 & x_3 & x_4 & x_5\\ \hline
  -9 & -2.5 & 0 & 0.25 & 0 & 0\\ \hline
  0.5 & -0.125 & 1 & 0.125& 0 & 0\\
  3 & -2.5 & 0 & 0.5 & 1 & 0\\
  0.5 & -1.625 & 0 & -0.375 & 0 & 1\\\hline
\end{array}
$$
Now, the only negative reduced cost is $\bar{c}_1$. When we try to bring $x_1$ into the basis, we see that all the $u$'s are negative, and thus the optimal cost is $- \infty$.

Therefore, with the following selected values, the optimal cost is $-\infty$:

$$
\alpha=-2, \beta=2, \delta=-2, \eta=8, \gamma=-2
$$

\subsection*{c}
To show that the current solution is feasible, we must fill in the remaining values such that $\beta \geq 0$; this will satisfy the non-negativity constraints. Filling in the remaining values should satisfy the equality constraints.

To show that the current solution is suboptimal, we must show that a pivot creates a solution with a smaller cost. To avoid the complexity of degeneracy, we first restrict $\beta$ to be a positive number (say 2), we obtain a feasible but not optimal solution. We can choose any random values for the other unknowns. Thus we use the below tableau,


$$
\begin{array}{c}
\\
 \\
x_3 \\
x_4 \\ 
x_5 \\
\end{array}
\begin{array}{|c|ccccc|}
    \hline
    & x_1 & x_2 & x_3 & x_4 & x_5\\ \hline
  -10 & 10 & -2 & 0 & 0 & 0\\ \hline
  4 & -1 & 10 & 1 & 0 & 0\\
  1 & -5 & -4 & 0 & 1 & 0\\
  2 & 1 & 3 & 0 & 0 & 1\\\hline
\end{array}
$$
With these choices, we can only pivot by letting $x_2$ enter the basis. In this case, we note that $\theta^*=\frac{x_{B(1)}}{u_1}$, and $x_3$ leaves the basis. This yields the tableau

$$
\begin{array}{c}
\\
 \\
x_2 \\
x_4 \\ 
x_5 \\
\end{array}
\begin{array}{|c|ccccc|}
    \hline
    & x_1 & x_2 & x_3 & x_4 & x_5\\ \hline
  -9.2 & 9.8 & 0 & 0.2 & 0 & 0\\ \hline
 0.4 & -0.1 & 1 & 0.1 & 0 & 0\\
 2.6 & -5.4 & 0 & 0.4 & 1 & 0\\
  0.8 & 1.3 & 0 & -0.3 & 0 & 1\\\hline
\end{array}
$$

We have thus found a new feasible solution with lower cost, with the following selected values:

$$
\alpha=-5, \beta=2, \delta=10, \eta=10, \gamma=1
$$