\subsection*{Part a}

Let $x$ be the basic feasible solution corresponding to the given tableau, with $x_{m+1}=x_n=0$ as $N=\{m+1, m+2, \cdots , n-1, n\}$ are the nonbasic indices. 

Suppose that, by iterating once (i.e. performing the pivot described in the problem) we reach the basic feasible solution $\bar{x}$. As the pivot described is nondegenerate, and is in a direction of decreasing cost, we know that $c'x>c'\bar{x}$. But, in general, the simplex method will iterate by providing a new basic feasible solution with the same or smaller cost, therefore any further iteration to some basic feasible solution $\tilde{x}$ will have the same or less cost than $\bar{x}$. But, since $c'x>c'\bar{x}$, we have $c'x>c'\tilde{x}$

We know that $x_n$ enters the basis at $\bar{x}$; suppose it then leaves the basis as we iterate to the basic feasible solution $\tilde{x}$. We consider the vector $d=(\tilde{x}-x)$, and the associated cost $c'd$. Since $c'x>c'\tilde{x}$, we have

$$
c'd=c'(\tilde{x}-x)=c'\tilde{x}-c'x<0
$$

However, since we know the reduced costs at $x$, we can also write $c'd$ as

$$
c'd=\sum_{i\in N}\bar{c}_id_i=\sum_{i=m+1}^n\bar{c}_id_i=\sum_{i=m+1}^{n-1}\bar{c}_id_i+\bar{c}_nd_n
$$

The problem tells us that $\bar{c}_i\geq 0$ for $i=m+1, m+2, \cdots, n-1$. Furthermore, since $x_i=0$ (as $i\in N$) and $\tilde{x}_i\geq 0$ (since $\tilde{x}$ is feasible), we know that $d_i\geq 0$ for all $i=m+1, m+2, \cdots, n-1$. Thus $\sum_{i=m+1}^{n-1}\bar{c}_id_i\geq 0$.

Furthermore, the problem tells us that $\bar{c}_n<0$, but since $x_n$ is nonbasic both at $x$ and at $\tilde{x}$, we know that $d_n=0$, and $\bar{c}_nd_n=0$. Thus

$$
c'd=\sum_{i=m+1}^{n-1}\bar{c}_id_i+\bar{c}_nd_n=\sum_{i=m+1}^{n-1}\bar{c}_id_i \geq 0
$$

However, this would suggest that the cost increases as we travel from $x$ to $\tilde{x}$, which it must not. Thus, we have reached a contradiction, and $x_n$ cannot enter the basis in further iterations of the simplex method.

Q.E.D.

\subsection*{Part b}

Consider the following tableau:

$$
\begin{array}{c}
\\
x_1 \\
x_2 \\ 
\end{array}
\begin{array}{|c|cccc|}
\hline
  0 & 0 & 0 & 0 & -1 \\ \hline
  0 & 1 & 0 & 1 & 1 \\
  1 & 0 & 1 & -2 & 0\\ \hline
\end{array}
$$


This tableau conforms to the assignment problem (where $n=4$):
\begin{itemize}
\item $m=2$: We have basic variables $x_1, x_2$
\item $\bar{c}_j\geq 0$ for $j=3, 4$
\item $\bar{c}_n<0$: $\bar{c}_4=-1$
\end{itemize}
Since $\bar{c}_4$ is the only negative reduced cost, we have $x_4$ enter the basis, and we have a degenerative pivot as $\theta^*=\frac{x_1}{a_{1,n}}=0$. Thus, this tableau is an example of the given tableau template.

We then pivot on $x_4$, yielding the following tableau:

$$
\begin{array}{c}
\\
x_4 \\
x_2 \\ 
\end{array}
\begin{array}{|c|cccc|c}
\hline
  0 & 1 & 0 & 1 & 0 & R_0=R_0+R_1\\ \hline
  0 & 1 & 0 & 1 & 1 & R_1=R_1\\
  1 & 0 & 1 & -2 & 0 & R_2=R_2\\ \hline
\end{array}
$$

At this point, the simplex method terminates, as all reduced costs are non-negative, and we have the optimal basic feasible solution $x=(0, 1, 0, 0)$ with optimal cost $0$. Note, however, that since the original tableau had the same cost, that basic feasible solution is also optimal, and in that solution, \textit{$x$ was nonbasic}. Thus, when the pivot is degenerate, $x_n$ need not be basic in an optimal basic feasible solution.

Q.E.D.