\textit{Note : This is Question 5 in the assignment sheet}

Consider an LP problem \textit{LP1} given by
\begin{align}
    \text{min. } &c'x \notag\\
    \text{S.T. } &Ax = b \notag\\
    &x_i \le U, i = 1,2....m \notag\\
    &x_i \ge 0, i = 1,2....m \notag\\
\end{align}

To transform to this to an equivalent one that contains the constraint $\sum_{i=1}^{m}x_i = 1$, we use the following: 
\[ x_i \le U\]
\[ \sum_{i=1}^{m}x_i \le mU \]
\[ \frac{\sum_{i=1}^{m}x_i}{mU} \le 1\]

Thus, we introduce a slack varibale $x_{m+1}$ such that, 
\[ \frac{\sum_{i=1}^{m}x_i}{mU} + x_{m+1} = 1\]

We then formulate a new LP problem \textit{LP2}, where

\[ x_{new} = [\frac{x_i}{mU} | x_{m+1}]\]
where $i=1,2....m$
Additionally, we modify the other matrices and vectors as 

\[ A_{new} = [A | 0]'\]
\[ c_{new} = [mU.c | 0]'\]
\[ b_{new} = \frac{1}{mU}.b\]

Thus \textit{LP2} is formulated as:


\begin{align}
    \text{min. } &c_{new}'x_{new} \notag\\
    \text{S.T. } &A_{new}x_{new} = b_{new} \notag\\
    &\sum_{i=1}^{m+1}x_{new,i} = 1 \notag \\
    &x_{new,i} \ge 0, i = 1,2....m+1 \notag\\
\end{align}

\textit{LP1} and \textit{LP2} are equivalent because they only differ by a slack variable. From Examples 1.4 and 1.5 in the slides, we know that introduction of a slack variable leads to an equivalent LP problem. Additionally, \textit{LP1} can always be constructed by projecting and scaling \textit{LP2} and the cost vector scales as well. Thus they must be equivalent.