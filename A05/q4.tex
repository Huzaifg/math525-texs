Consider an LP problem \textit{LP1} given by
\begin{align}
    \text{min. } &c'x \notag\\
    \text{S.T. } &Ax = b \notag\\
    &x_i \le U, i = 1,2....m \notag\\
    &x_i \ge 0, i = 1,2....m \notag\\
\end{align}

To transform to this to an equivalent one that contains the constraint $\sum_{i=1}^{n}x_i = 1$, we use the following: 
\[ x_i \le U\]
\[ \sum_{i=1}^{m}x_i \le mU \]
\[ \frac{\sum_{i=1}^{m}x_i}{mU} \le 1\]

Thus, we introduce a slack varibale $x_{m+1}$ such that, 
\[ \frac{\sum_{i=1}^{m}x_i}{mU} + x_{m+1} = 1\]

We then formulate a new LP problem \textit{LP2}, where

\[ x_{new} = (\frac{x_i}{mU} | x_{m+1})\]
where $i=1,2....m$
Additionally, we modify the other matrices and vectors as 

\[ A_{new} = [A | 0]'\]
\[ c_{new} = [mU\cdot c | 0]'\]
\[ b_{new} = \frac{b}{mU}\]

Thus \textit{LP2} is formulated as:


\begin{align}
    \text{min. } &c_{new}'x_{new} \notag\\
    \text{S.T. } &A_{new}x_{new} = b_{new} \notag\\
    &\sum_{i=1}^{m+1}x_{new,i} = 1 \notag \\
    &x_{new,i} \ge 0, i = 1,2....m+1 \notag\\
\end{align}

To show that \textit{LP1} and \textit{LP2} are equivalent problems, we show that for every feasible solution in either of the problems we can find a solution in the other problem with the same or lower cost.\\

Let $x$ be a feasible solution to \textit{LP1}, and consider the following solution to \textit{LP2}:

\[x_{new} = \left[\frac{x_i}{mU} \:|\: 1 - \frac{\sum_{i=1}^{m}x_i}{mU}\right]'\]

Firstly, because the last column of $A_{new}=0$, we know that $x_{new, m+1}$ will not factor into any of the constraints. Thus we have $A_{new}x_{new}=\frac{1}{mU}Ax=\frac{b}{mU}=b_{new}$, and we satisfy the constraints.

Secondly, by construction, we know that $\sum_{i=1}^{m+1}x_i=\sum_{i=1}^{m}\frac{x_i}{mU}+(1 - \frac{\sum_{x_i}^{m}}{mU})=1$, and we satisfy that constraint.

Thirdly, because $x_i, m, U$ are non-negative, $x_{new}\geq 0$, and $x_{new}$ satisfies that constraint.

Finally, since $c_{new}'x_{new}=\sum_{i=1}^m mU\cdot c_i\cdot \frac{x_i}{mU} + 0\cdot (1 - \frac{\sum_{i=1}^{m}x_i}{mU})=c'x$, we have a solution with the same cost.


Similarly, let $x_{new}$ be a feasible solution in \textit{LP2}, then we just drop $x_{m+1}$ and project from $\mathbb{R}^{m+1} \text{ to } \mathbb{R}^m$ to get

\[ x = (mUx_{new,1}, mUx_{new, 2}, \cdots, mUx_{new, m})\]
To make life easier, we note that the above arguments can be applied in reverse to show that $x$ has the same cost as $x_{new}$. Thus, the two problems are equivalent.

We have shown so far that we can turn an $m$-variable problem into a $(m+1)$-variable problem where $\sum_{i=1}^{m+1}x_i=1$. To obtain a problem including the constraint $\sum_{i=1}^nx_i=1$, we simply repeat the procedure $n-m$ times on the output of the previous iteration to produce a problem with $n$ variables.

Q.E.D.