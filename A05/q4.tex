Consider an LP problem \textit{LP1} given by
\begin{align}
    \text{min. } &c'x \notag\\
    \text{S.T. } &Ax = b \notag\\
    &x_i \le U, i = 1,2....n \notag\\
    &x_i \ge 0, i = 1,2....n \notag\\
\end{align}

To transform to this to an equivalent one that contains the constraint $\sum_{i=1}^{n}x_i = 1$, we use the following: 
\[ x_i \le U\]
\[ \sum_{i=1}^{n}x_i \le nU \]
\[ \frac{\sum_{i=1}^{n}x_i}{nU} \le 1\]

Thus, we introduce a slack variable $x_{n+1}$ such that
\[ \frac{\sum_{i=1}^{n}x_i}{nU} + x_{n+1} = 1\]
\[ x_{n+1}=1-\frac{\sum_{i=1}^{n}x_i}{nU}\]

We then formulate a new LP problem \textit{LP2}, where

\[ x_{new} = \begin{pmatrix}\frac{x_1}{nU} & \frac{x_2}{nU}&\cdots & \frac{x_n}{nU} & x_{n+1}\end{pmatrix}\]
where $i=1,2....n$
Additionally, we modify the other matrices and vectors as 

\[ A_{new} = [A | 0]\]
\[ c_{new} = \begin{pmatrix}nUc_1 & nUc_2 & \cdots & nUc_n & 0\end{pmatrix}\]
\[ b_{new} = \frac{b}{nU}\]

Thus \textit{LP2} is formulated as:


\begin{align}
    \text{min. } &c_{new}'x_{new} \notag\\
    \text{S.T. } &A_{new}x_{new} = b_{new} \notag\\
    &\sum_{i=1}^{n+1}x_{new,i} = 1 \notag \\
    &x_{new,i} \ge 0, i = 1,2....n+1 \notag\\
\end{align}

Note the constraint that all of the variables sum to one, which was the goal of this exercise.\\

To show that \textit{LP1} and \textit{LP2} are equivalent problems, we show that for every feasible solution in either of the problems we can find a solution in the other problem with the same or lower cost.\\

Let $x$ be a feasible solution to \textit{LP1}, and consider the following solution to \textit{LP2}:

\[x_{new} = \begin{pmatrix}\frac{x_1}{nU} & \frac{x_2}{nU} & \cdots & \frac{x_n}{nU} & (1 - \frac{\sum_{i=1}^{n}x_i}{nU})\end{pmatrix}\]

Firstly, because the last column of $A_{new}=0$, we know that $x_{new, n+1}$ will not factor into any of the constraints. Thus we have $A_{new}x_{new}=\frac{1}{nU}Ax=\frac{b}{nU}=b_{new}$, and we satisfy the constraints.

Secondly, by construction, we know that $\sum_{i=1}^{n+1}x_i=\sum_{i=1}^{n}\frac{x_i}{nU}+(1 - \frac{\sum_{x_i}^{n}}{nU})=1$, and we satisfy that constraint.

Thirdly, because $x_i, n, U$ are non-negative, $x_{new}\geq 0$, and $x_{new}$ satisfies that constraint.

Finally, since $c_{new}'x_{new}=\sum_{i=1}^n nU\cdot c_i\cdot \frac{x_i}{nU} + 0\cdot (1 - \frac{\sum_{i=1}^{n}x_i}{nU})=c'x$, we have a solution with the same cost.


Similarly, let $x_{new}$ be a feasible solution in \textit{LP2}, then we just drop $x_{n+1}$ and project from $\mathbb{R}^{n+1} \text{ to } \mathbb{R}^n$ to get

\[ x = (nUx_{new,1}, nUx_{new, 2}, \cdots, nUx_{new, m})\]
To make life easier, we note that the above arguments can be applied in reverse to show that $x$ has the same cost as $x_{new}$. Thus, the two problems are equivalent, and we have shown that we can turn an $n$-variable problem into an equivalent problem where $\sum_{i=1}^{n}x_i=1$.

Q.E.D.