Consider that during Phase 1 of the simplex algorithm a artificial variable leaves the basis and becomes non-basic. We can then write this as two LP problems, 

LP1 before the artificial leaves the basis (consider m actual variables and n artificial variables):

\begin{align}
    \text{min. } &\sum_{i=1}^{m-n}y_i \notag\\
    \text{S.T. } &Ax + y = b \notag\\
    &x_i \ge 0, i = 1,2....m \notag\\
    &y_i \ge 0, i = 1....m-n \notag\\
\end{align}

LP2 after the artificial variable leaves the basis and we drop it out:

\begin{align}
    \text{min. } &\sum_{i=1}^{m-n-1}y_i \notag\\
    \text{S.T. } &Ax + y = b \notag\\
    &x_i \ge 0, i = 1,2....m \notag\\
    &y_i \ge 0, i = 1....m-n-1 \notag\\
\end{align}

Now, we need to show that LP2 is feasible if and only if LP1 is feasible. Also, we need to show that if LP1 and LP2 both have the same optimal cost of 0, if they are feasible. 
If LP1 is feasible, there exists an $x$ such that with $y=0$ we have $(x,y)$ to be a feasible solution. This feasible solution has zero cost. Note that this solution is also feasible for LP2 with the vector $(x,y)$ just being of lower dimension and with the same optimal cost of 0. For the inverse, if $(x,y)$ is a vector of size $n-1$ and is feasible in LP2, then there is a corresponding feasible solution $(x,y,0)$ in LP1 with the same cost. \\

Now, if LP1 is infeasible, then there exists no $(x,0)$ in LP1 that is feasible. If follows then that there is no $(x,0)$, where the zero vector has a dropped dimension, in LP2 that is feasible. \\

To summarize, even if an artificial variable is dropped from the tableau once it becomes non-basic, it does not affect the outcome of phase 1. 
